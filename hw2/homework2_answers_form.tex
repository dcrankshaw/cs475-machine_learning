\documentclass[letterpaper,11pt]{article}
\title{CS475 Machine Learning, Fall 2012: Homework 2}
\author{\bf Daniel Crankshaw - dcranks1}


\usepackage[margin=1in]{geometry}
% \usepackage{hyperref}
\usepackage[colorlinks]{hyperref}
\usepackage{capt-of}
\usepackage{amssymb}
\usepackage{amsmath}
\usepackage{url}
\usepackage{graphicx}
\usepackage{color}
\usepackage{bbm}
\usepackage{float}
\usepackage{graphicx}
\usepackage{wrapfig}
\usepackage{url}
\usepackage{wrapfig}
\usepackage{hyperref} 
\usepackage{color}
\usepackage{amstext}
\usepackage{enumerate}
\usepackage{amsmath,bm}
\usepackage{fullpage}

\def\P{\mathbb{P}}
\def\E{\mathbb{E}}
\def\R{\mathbb{R}}

\renewcommand{\baselinestretch}{1.15}    

\begin{document}

\maketitle

%\paragraph*{Instructions} Please fill in your name and JHED ID above.

\paragraph{Question 1:}
\begin{enumerate}[(a)]
\item $P = \left(\frac{1}{2}\right)^3 = \frac{1}{8}$
\item Let H be a random variable representing the number of heads flipped in 10 coin
tosses.\\
$P(H \geq 3) = \sum\limits_{i=3}^{10} P(H=i) = \sum\limits_{i=3}^{10} \binom{10}{i}
= 968$.
\item It takes $14$ flips on average.
\end{enumerate}

\paragraph{Question 2:}
\begin{enumerate}[(a)]
\item
\begin{align}
\mathbb{P}(X) & = \sum\limits_{j=1}^k \mathbb{P}(X|C=j) \mathbb{P}(C=j)
\end{align}
\item
  \begin{align}
    \E(X) & = \E[\E(X | C)]\\
          & = \sum\limits_{j=1}^k \E(X | C=j) \P(C=j)\\
          & = \sum\limits_{j=1}^k \mu_j \pi_j
\end{align}
\item
  \begin{align}
  log (\P{D |\mu_j, \sigma_j for j=1,\ldots,k}) & = \sum\limits_{i=1}^{n} log(\P(x_n | \mu_j, \sigma_j))\\
                                                & = \sum\limits_{i=1}^n log \left\(\sum\limits_{j=1}^k \P(X=x_i|C=j) \P(C=j) \right\)\\
                                                & = \sum\limits_{i=1}^n log\left[\sum\limits_{j=1}^k \pi_j \frac{1}{\sigma_j\sqrt{2\pi}}\exp\left\{{-\frac{(x_i-\mu_j)^2}{2\sigma_j^2}}\right\} \right]\\
  \end{align}

\end{enumerate}

\paragraph{Question 3:}
First I will prove the first part of the set of inequalities, and then the second part.
For
\begin{equation}
  \|y-X_1\hat{\beta}_3\|_2^2 \geq \|y-X_1\hat{\beta}_0\|_2^2
\end{equation}
Assume that $\|y-X_1\hat{\beta}_3\|_2^2 < \|y-X_1\hat{\beta}_0\|_2^2$. Then there exists some $\beta_0' = \hat{\beta}_3$ 
such that $\|y-X_1\beta_0'\|_2^2 < \|y-X_1\hat{\beta}_0\|_2^2$. Thus, $\hat{\beta}_0$ is not the argmin. But $\hat{\beta}_0$
is the argmin, and so we have reached a contradiction. Therefore, $\|y-X_1\hat{\beta}_3\|_2^2 \geq \|y-X_1\hat{\beta}_0\|_2^2$.


\begin{equation}
  \|y-X_1\hat{\beta}_0\|_2^2 \geq \|y-X_1\hat{\beta}_1-X_2\hat{\beta}_2\|_2^2
\end{equation}
Notice that $\mathop{\rm argmax}_{\beta_0} X_1\beta_0 \geq \textbf{0}$ (where $\textbf{0}$ is the n-dimensional null vector). Similarly,
$\mathop{\rm argmax}_{\beta_1} X_1\beta_1 \geq \textbf{0}$ and $\mathop{\rm argmax}_{\beta_2} X_2\beta_2 \geq \textbf{0}$.
And because we are subtracting each of these quantities from $y$, no matter what $X_1\hat{\beta}_0$ is, we can always get the same value
from $X_1\hat{\beta}_1$. And because $\mathop{\rm argmax}_{\beta_2} X_2\hat{\beta}_2 \geq \textbf{0}$ then we either have
$X_2\hat{\beta}_2 = \textbf{0}$ in which case the two expressions are equal, or 
$X_2\hat{\beta}_2 > \textbf{0}$ in which case the $\|y-X_1\hat{\beta}_0\|_2^2 > \|y-X_1\hat{\beta}_1-X_2\hat{\beta}_2\|_2^2$. Thus the inequality
holds true.

\paragraph{Question 4:}
\begin{enumerate}[(a)]
\item With non-overlapping samples, in the worst case every split exactly divides the data in half. If we divide the data in half
  at every split, then each sample needs $log_2 n$ splits to be labeled. Because the samples are non-overlapping, there will never
  be a split that contains data with the same set of features with different labels. Thus, we can always perform a series of feature
  splits that will result in a node in the decision tree with all the labels agreeing. Therefore, we can perform a perfect labeling.
  Therefore, in this situation, we can always perform a perfect labeling of all $n$ samples with a decision tree of depth at most
  $log_2 n$.
\item We cannot. If we lose the assumption that all labels are non-overlapping, then we can have a situation where we have at
  least two samples with the exact same feature vector that have different labels. This situation means that even after $log_2 n$
  feature splits (isolating all samples with this feature vector from any with distinct feature vectors) we will have a set
  of data at a node which do not all have the same label. Therefore, we cannot perfectly label the data.
\end{enumerate}

\paragraph{Question 5:}
\begin{enumerate}[(a)]
\item Convex formulations have exactly one minimum, meaning that the global minimum is also the only local minimum. This makes
  it much easier to find the minimum of the function.
\item
  The definition of convexity is as follows. A function $f(x)$ is convex if for any two points $x_1$, $x_2$ in $X$ and $t \in [0,1]$
  \begin{equation}
    f(t x_1 + (1 - t) x_2) \leq t f(x_1) + (1-t) f(x_2)
  \end{equation}
  \begin{align}
    h(x) & = f(x) + g(x)\\
    h(t x_1 + (1 - t) x_2) & = f(t x_1 + (1 - t) x_2) + g(t x_1 + (1 - t) x_2)\\
    h(t x_1 + (1 - t) x_2) & \leq t f(x_1) + t g(x_1) + (1-t)f(x_2) + (1-t) g(x)2 & (\text{by the definition of convexity})\\
                           & = t(f(x_1) + g(x_1)) + (1-t)(f(x_2) + g(x_2))\\
                           & = th(x_1) + (1-t)h(x_2)
  \end{align}
  Therefore, $h(t x_1 + (1 - t) x_2) \leq th(x_1) + (1-t)h(x_2)$ and thus satisfies the definition of a convex function.
\item $f'(x) = x^4$ and $g'(x) = x^2$ results in the a non-convex function $f'(x) - g'(x)$.\\
  $f(x) = e^x$ and $g(x) = 2^x$ results in a convex function $f(x) - g(x)$.
\end{enumerate}

\end{document}


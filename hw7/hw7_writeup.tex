\documentclass[letterpaper,11pt]{article}
\title{CS475 Machine Learning, Fall 2012: Homework 7}
\date{}
\author{\bf Daniel Crankshaw}


\usepackage[margin=1in]{geometry}
% \usepackage{hyperref}
\usepackage[colorlinks]{hyperref}
\usepackage{capt-of}
\usepackage{amssymb}
\usepackage{amsmath}
\usepackage{url}
\usepackage{graphicx}
\usepackage{color}
\usepackage{bbm}
\usepackage{float}
\usepackage{graphicx}
\usepackage{wrapfig}
\usepackage{url}
\usepackage{wrapfig}
\usepackage{hyperref} 
\usepackage{color}
\usepackage{amstext}
\usepackage{enumerate}
\usepackage{amsmath,bm}
\usepackage{fullpage}
    
\renewcommand{\baselinestretch}{1.15}    

\begin{document}

\maketitle

\paragraph{Question 1:}
\subparagraph{Bayesian Network}
\begin{enumerate}[(a)]
\item
    {\bf A} and {\bf B} are d-separated in this example. The set of paths going through $x_5$ is blocked
    by the tail to tail intersection at $x_5$, which is in set {\bf C}. The set of paths going through
    $x_{14}$ is blocked by the head to tail intersection at $x_{14}$ because $x_{14}$ is in {\bf C}.
\item
    No the sets are not d-separated.
\item
    Yes. The sets are d-separated because every path between {\bf A} and {\bf B} must pass through
    $x_{15}$ with a head to head intersection. And neither $x_{15}$ nor any of its descendants are
    in {\bf C}, so that node blocks.
\item
    No. The sets are not d-separated. All the head to head intersections of the paths have $x_{15}$
    as a descendant (or occur at $x_{15}$) and so none of those nodes block. And some of the paths
    do not go through any other nodes in {\bf C} to meet any of the tail to tail or head to tail
    blocking conditions.
\end{enumerate}
\subparagraph{Markov Random Field}
\begin{enumerate}[(a)]
\item
    {\bf A} and {\bf B} are d-separated in this example because all paths must go through
    $x_{5}$ or $x{14}$ which are in {\bf C} and thus blocked.
\item
    Yes the sets are d-separated because all paths must go through $x_{15}$ which is in set {\bf C}
    and thus blocks all paths.
\item
    No because for example the path $x{4} -> x{6} -> x{11} -> x{15} -> x{12} -> x{8} -> x{5}$ contains
    no nodes in {\bf C} and is therefore unblocked.
\item
    Yes the sets are d-separated because all paths must go through $x_{15}$ which is in set {\bf C}
    and thus blocks all paths.
\end{enumerate}

\paragraph{Question 2:}
    This question is essentially asking us for the Markov Blanket for node $x_2$. That is the set of nodes
    \{ $x_5$ \}.

\paragraph{Question 3:}


\paragraph{Question 4:}
\begin{enumerate}[(a)]
\item
\item
\end{enumerate}

\paragraph{Question 5:}
\begin{enumerate}[(a)]
\item
\item
\item
\end{enumerate}

\end{document}

